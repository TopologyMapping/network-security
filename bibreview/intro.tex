% LTeX: language=pt-BR

\section{Introdução}\label{sec:intro}

Vulnerabilidades de software podem afetar negativamente o funcionamento de sistemas computacionais, por exemplo, levando à degradação de desempenho, indisponibilidade, corrupção ou vazamento de dados, e uso não autorizado de recursos~\cite{ghaffarian17vulnsurvey, gonzalez19vuln}.
Incidentes de segurança trazem bilhões de prejuízo e afetam milhões de usuários todos os anos~\cite{gartner18forecast, gartner18forecast, zscaler23report, alanazi2023scada, defense16ukraine, hill18heist}.

A comunidade de segurança reporta, verifica e cataloga novas vulnerabilidades continuamente~\cite{smyth17vuln}. Ferramentas de monitoramento proprietárias e de código aberto implementam testes para detectar e identificar um número crescente de vulnerabilidades. Devido à complexidade, dinamicidade e alto nível de integração de sistemas computacionais modernos, a execução destas ferramentas geralmente identifica uma grande quantidade de vulnerabilidades, frequentemente de severidade crítica ou alta~\cite{smyth17vuln}. A alta taxa de vulnerabilidades identificadas frequentemente sobrecarrega equipes de desenvolvedores, operadores e mesmo analistas de segurança, inviabilizando que entidades consigam investigar vulnerabilidades conhecidas em tempo hábil, muito menos implementar ou aplicar contramedidas. Esta situação é agravada quando entidades não conseguem alocar recursos humanos para a resolução de vulnerabilidades. Mesmo quando conseguem, alocar os limitados recursos humanos disponíveis de forma eficiente é desafiador face à quantidade de vulnerabilidades~\cite{ethembabaoglu2024unpatchables, smale23firehose}.

Um ponto positivo neste cenário é que uma fração significativa das vulnerabilidades reportadas possuem mitigações (\emph{patches}) no momento em que são tornadas públicas~\cite{smyth17vuln}. Idealmente isto permitiria que equipes de analistas mantivessem sistemas atualizados e seguros contra a maioria das vulnerabilidades. Porém, na prática, equipes de segurança encontram diversos desafios em atingir este objetivo. A manutenção de informações atualizadas sobre vulnerabilidades requer atenção de especialistas, o que leva a várias vulnerabilidades terem informações incompletas ou desatualizadas~\cite{gonzalez19vuln}. A complexidade de vulnerabilidades de segurança aumenta junto da complexidade dos sistemas afetados~\cite{le2021survey}. A aplicação de atualizações requer acompanhamento por um analista, precisa ser realizada sem comprometer o funcionamento dos sistemas vulneráveis, e pode exigir ajustes quando as atualização não são retro-compatíveis (\emph{backwards compatible})~\cite{panesar23iot, thomas2020catch}. Combinados com a grande quantidade de vulnerabilidades detectadas, a realidade é que sistemas continuam vulneráveis mesmo quando atualizações estão disponíveis~\cite{spring2023analysis, ethembabaoglu2024unpatchables, smale23firehose}.

Um significativo esforço vem sendo realizado pela comunidade científica, por instituições especializadas, e empresas privadas para tentar remediar esta situação. Nesta revisão bibliográfica focamos em esforços nestas três frentes que visam priorizar vulnerabilidades de software com o objetivo de auxiliar equipes de desenvolvedores e analistas manter sistemas seguros considerando diversas restrições impostas, por exemplo, por disponibilidade de recursos humanos, hardware, software, clientes e usuários.

O conhecimento adquirido na construção deste relatório irá orientar o desenvolvimento de novas soluções para priorização de vulnerabilidades. Neste contexto, temos três ideias que esperamos irão permitir a construção de soluções mais eficazes.  Primeiro, vulnerabilidades distintas têm impacto variado dependendo da entidade afetada. Por exemplo, uma bancos e empresas de análise de crédito podem priorizar vulnerabilidades que permitem vazamento de dados e afetam a privacidade de seus clientes ou que comprometam a integridade de bases de dados, enquanto uma empresa de distribuição de conteúdo pode priorizar vulnerabilidades que afetam o desempenho ou disponibilidade do serviço. Segundo, diferentes equipes de funcionários podem solucionar vulnerabilidades de diferentes formas, e com diferentes graus de eficiência. Por exemplo, um time de desenvolvedores pode melhorar a segurança atualizando a pilha de software em uso para versões mais recentes, enquanto um time de administradores de sistema terá maior facilidade em mitigar vulnerabilidades configurando \emph{firewalls} ou políticas de acesso (ACLs). Por último, contexto externo à vulnerabilidade pode (des)motivar esforços imediatos de remediação. Vulnerabilidades recentes, de fato exploradas, com provas de conceito ou em sistemas críticos devem ser tratadas com prioridade visto a elevada chance de serem alvo de ataques e danos potenciais. Por exemplo, uma vulnerabilidade de \emph{cross-site scripting} (XSS)~\cite{hydara15xss} ou de \emph{SQL injection}~\cite{nasereddin23sqlinject} requer remediação imediata: Estes ataques são comuns, exploráveis remotamente, amplamente visados e ativamente procurados por sistemas de varreduras (\emph{scanning}). Por outro lado, uma vulnerabilidade que requer acesso de administrador ou acesso à rede local pode receber uma prioridade menor em função da dificuldade de ser explorada.

Nosso objetivo final é empoderar entidades e seus funcionários, direcionando esforços em cibersegurança para que consigam mitigar um número maior de vulnerabilidades mais relevantes, obtendo a máxima melhoria de segurança dentro dos recursos dedicados. Pretendemos gerar contribuições práticas desenvolvendo e empacotando as soluções desenvolvidas para facilitar seu uso e implantação.
